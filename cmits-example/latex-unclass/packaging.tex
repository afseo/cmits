% --- BEGIN DISCLAIMER ---
% Those who use this do so at their own risk;
% AFSEO does not provide maintenance nor support.
% --- END DISCLAIMER ---
% --- BEGIN AFSEO_DATA_RIGHTS ---
% This is a work of the U.S. Government and is placed
% into the public domain in accordance with 17 USC Sec.
% 105. Those who redistribute or derive from this work
% are requested to include a reference to the original,
% at <https://github.com/afseo/cmits>, for example by
% including this notice in its entirety in derived works.
% --- END AFSEO_DATA_RIGHTS ---
\chapter{Packaging}
\label{Packaging}

You should put software in packages where possible. This chapter discusses
how and why, in general terms. How this works out in your organization
will vary.



\section{Why package?}

Packaging software makes it easier to add, remove and upgrade. It also can
push the work of satisfying software dependencies off of you, the
administrator, and onto the packaging system. Software that's been
packaged and installed is on the local hard drive of each machine, so it
works just as well when the network is gone (on the laptop of someone who
is on a business trip, for example), and runs faster. It's easier to
control the interactions between software providing some duplicate
functionality (\emph{e.g.}, OpenMPI vs.  MPIch2) when it's in
packages---if it's not installed it's unavailable, and if you want a
per-machine or per-user choice, the \emph{alternatives} subsystem or the
\emph{modules} subsystem can help you to make that choice completely and
simply.



\section{The RPM package manager}

We speak here about packaging in the context of \emph{RPM}, the RPM
Package Manager (formerly Red Hat Package Manager). RPM supports
installation, removal and upgrade of packages of software, and keeps track
of data about packages which eases administration, such as which packages
depend on which others, whether a package has been cryptographically
signed, what versions of packages are installed, and whether files which
have been installed as part of a package have changed since being
installed.

Before packaging a piece of software you will want to see if someone else
has packaged it already and if that package is suitable. Fedora's EPEL
(Extra Packages for Enterprise Linux) project
(\url{http://fedoraproject.org/wiki/EPEL}) packages some software not
packaged as part of Red Hat Enterprise Linux. When obtaining and
installing RPM packages not from the vendor, you should make sure you
trust the packager. Owing to RPM's flexibility and use across several
distributions of Linux, random RPMs you find on the Internet will not
necessarily install or run properly on Red Hat Enterprise Linux.

If a package is not already extant for the software you need, you can make
your own package. The act of packaging software with RPM is usually almost
as easy as installing it from source. See the Fedora RPM Guide
(\url{http://docs.fedoraproject.org/drafts/rpm-guide-en/}) for more about
the generalities of this topic; specifics will vary by organization.


% --- BEGIN DISCLAIMER ---
% Those who use this do so at their own risk;
% AFSEO does not provide maintenance nor support.
% --- END DISCLAIMER ---
% --- BEGIN AFSEO_DATA_RIGHTS ---
% This is a work of the U.S. Government and is placed
% into the public domain in accordance with 17 USC Sec.
% 105. Those who redistribute or derive from this work
% are requested to include a reference to the original,
% at <https://github.com/afseo/cmits>, for example by
% including this notice in its entirety in derived works.
% --- END AFSEO_DATA_RIGHTS ---
% This is included by the Packaging chapter.

\section{Organization-specific details}

You should write your own organization-specific guidelines for how to
package software, how to track and control changes to your organization's
custom packages, and how to deploy packaged software.

