% --- BEGIN DISCLAIMER ---
% Those who use this do so at their own risk;
% AFSEO does not provide maintenance nor support.
% --- END DISCLAIMER ---
% --- BEGIN AFSEO_DATA_RIGHTS ---
% This is a work of the U.S. Government and is placed
% into the public domain in accordance with 17 USC Sec.
% 105. Those who redistribute or derive from this work
% are requested to include a reference to the original,
% at <https://github.com/afseo/cmits>, for example by
% including this notice in its entirety in derived works.
% --- END AFSEO_DATA_RIGHTS ---
\chapter{Introduction}
\label{Introduction}

This document is a record of how a number of computers are configured and
maintained. 

Many of the elements of this policy are motivated by requirements in
higher-level policies, such as Department of Defense (DoD) Instruction
8500.2 Information Assurance (IA) Controls \cite{dodi-8500-2}, the Defense
Information Services Agency's (DISA) UNIX Security Requirements Guide
(SRG, \cite{unix-srg}), or various Air Force Instructions (AFIs).
\S\S\ref{iac-compliance-summary}, \ref{UNIXSRGCompliance},
\ref{SPANSTIGCompliance} and~\ref{DatabaseSTIGCompliance} show how we meet
those requirements.

Under this policy, \emph{hosts} (individual computers, real or virtual)
are configured using \emph{Puppet}, an automated, policy-based
system configuration tool.  \S\S\ref{Contingency}~and~\ref{Production}
discuss how administrators can follow this policy to configure systems
manually in a contingency situation, or set Puppet in place to enforce the
policy automatically, as is usual in production.

The same documents that impose requirements on system configuration also
impose requirements on system administrators and users, about what to do
and how to do it. \S\ref{ProceduresForUsers} initiates users in their
responsibilities, and \S\ref{Procedures} discusses day-to-day tasks done
by administrators.

\S\S\ref{Maintenance}~and~\ref{Packaging} discuss how to maintain the
policy and this document, and how properly to automate the installation
and removal of software.

Finally, the policy (\S\ref{Policy}) and its attendant files
(\S\ref{AttendantFiles}) follow in all their detail.


\section{Typographical conventions}

Much of this document has to do with compliance. Near any statement of
status regarding compliance there is a margin note with the name of the
specific requirement. For example, in~\S\ref{class_ssh::fips}, there is
Puppet code which configures hosts for compliance with
\unixsrg{GEN000590}. Just before that code is a comment explaining what
the code does and how that complies with the requirement. Beside the code
and discussion, in the margin, is a note, ``GEN000590.'' Where
requirements are not applicable, the margin note looks like: ``N/A:
GEN000590.'' Where compliance is automated, the margin note looks like:
``auto: GEN000590.'' Where compliance is merely documented, the margin
note merely says, ``GEN000590.'' And where we are not yet compliant, the
margin note is red.

Section numbers are denoted with \S.


\section{Navigational aids}

Links between parts of this document abound; if you are viewing it as a
PDF file, you can click on any section number you see in the text to visit
that section. Your PDF reader may have a list of bookmarks in a sidebar,
by which you can easily skip around the document.  Even if it doesn't, you
should find the entire table of contents to be clickable. You may also
find the numerous indices, with their clickable page numbers, to be a
useful resource.


\section{Colophon}
\label{Colophon}

This document was automatically constructed on the date shown on the title
page, from the complete set of Puppet policy files (the \emph{manifest},
in Puppet parlance), which contain the policy as enforced on that date.
The motivating values behind this are that accuracy, completeness and
currency are more important than readability, editability, and
approachability.

Put another way, if this were a Word document written by hand, anybody
would know how to open it and edit it. But such a document would likely
not be up-to-date, because, while updating the policy would fix immediate
user problems, updating the Word document would only have vague future
benefits. That task could be easily skipped entirely in the near term, or
administrators might easily jot down changes without supplying enough
context or detail for their writings to be useful six months down the
road. As it is, when administrators edit this policy, the source of the
documentation is in the same file, on the same screenful of letters. If
an administrator updates part of the policy, and the comment right above
it becomes wrong, the juxtaposition makes that fact obvious, and more
likely to be rectified. The administrator is also more likely to write the
documentation in the first place, because no additional files or programs
must be opened in order to begin writing it.

See~\S\ref{Maintenance} for more information about how to generate this
finished document from its pieces, and how to manage and document changes
you make to the policy herein.
