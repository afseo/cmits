% --- BEGIN DISCLAIMER ---
% Those who use this do so at their own risk;
% AFSEO does not provide maintenance nor support.
% --- END DISCLAIMER ---
% --- BEGIN AFSEO_DATA_RIGHTS ---
% This is a work of the U.S. Government and is placed
% into the public domain in accordance with 17 USC Sec.
% 105. Those who redistribute or derive from this work
% are requested to include a reference to the original,
% at <https://github.com/afseo/cmits>, for example by
% including this notice in its entirety in derived works.
% --- END AFSEO_DATA_RIGHTS ---
\chapter{Database STIG compliance}
\label{DatabaseSTIGCompliance}

Some pieces of database software are included in RHEL and supported by Red
Hat. Because of this, many items of compliance with the Database STIG are
provided by the operating system, and others are controlled by this
\CMITSPolicy . These items are documented here, rather than in the
documents of whatever Automated Information System (AIS) may be using
the database, to avoid duplication and ensure accuracy.



\section{Database STIG compliance under PostgreSQL}
\label{DatabaseSTIGPostgreSQL}

PostgreSQL is included in RHEL. Some Database STIG requirements are
therefore met as part of the requirements placed on operating system
configuration and maintenance by other documents, like the UNIX SRG.
See~\S\ref{dbms-with-rhel} for details on these.

Some requirements are met by configuring PostgreSQL in a certain way.
See~\S\ref{module_postgresql} for details on these.

Many other requirements on DBMS configuration are met by the default
configuration of PostgreSQL as included in RHEL. These are documented in
this section.

\dinkus

\documents{iacontrol}{DCCS-1}\documents{databasestig}{DG0007}%
The Database STIG is the primary document used in securing the PostgreSQL
DBMS under RHEL.

\bydefault{RHEL5,RHEL6}{iacontrol}{ECLP-1}\bydefault{RHEL5,RHEL6}{databasestig}{DG0005}%
The DBA account for the PostgreSQL DBMS under RHEL is \verb!postgres!,
which does not have any ``host system administrator privileges.''

\bydefault{RHEL5,RHEL6}{iacontrol}{DCFA-1}\bydefault{RHEL5,RHEL6}{databasestig}{DG0014}%
PostgreSQL as distributed in RHEL contains no ``demonstration or sample
databases or applications.''

\bydefault{RHEL5,RHEL6}{iacontrol}{ECLP-1}\bydefault{RHEL5,RHEL6}{databasestig}{DG0040,DG0041}%
The ``DBMS software installation account'' for the PostgreSQL DBMS under
RHEL, \verb!postgres!, is not permitted to log in by default. Only system
administrators can perform actions using the privileges of this user, by
the use of the \verb!su! or \verb!sudo! commands; all uses of the account
are logged by default. (See the UNIX SRG Compliance index on
\unixsrg{GEN003660} and \unixsrg{GEN001060}.)

\notapplicable{iacontrol}{ECLP-1}\notapplicable{databasestig}{DG0051}%
PostgreSQL as included in RHEL does not include ``job queues managed by
the database.''

\notapplicable{iacontrol}{ECAN-1}\notapplicable{databasestig}{DG0053}%
PostgreSQL does not use a ``client database connection configuration
file.''

\documents{iacontrol}{IAIA-1}\documents{databasestig}{DG0067}%
For PostgreSQL as included in RHEL, the lists of ``DBMS database
objects, database configuration files, associated scripts and applications
defined within or external to the DBMS that access the database, and DBMS
/ user environment files/settings'' are as follows:

The list of system-level DBMS-related files can be obtained by running the
commands

\begin{verbatim}
rpm -ql postgresql-server
rpm -q --configfiles postgresql-server
\end{verbatim}

on a server with PostgreSQL installed. ``User environment files/settings''
are stored in the user's shell initialization file and \verb!.pgpass!
file. See~\S\ref{module_home} and~\S\ref{ProceduresForUsers} for more
details.

\documents{iacontrol}{IAIA-1}\documents{databasestig}{DG0068}%
The \verb!psql! command allows specification of a password on the command
line; this practice is strictly prohibited, as required,
in~\S\ref{ProceduresForUsers}.

\bydefault{RHEL5,RHEL6}{iacontrol}{ECRC-1}\bydefault{RHEL5,RHEL6}{databasestig}{DG0084}%
PostgreSQL has an \emph{auto-vacuuming} feature which ``clear[s] residual
data from storage locations after use.'' The default configuration
included in RHEL enables auto-vacuuming.

\unimplemented{databasestig}{DG0085}{We need to revisit DBA users in light
of later checklist requirements.}

\documents{iacontrol}{DCFA-1}\documents{databasestig}{DG0098}%
A review of the PostgreSQL 8.4 documentation has shown that PostgreSQL
does not support ``objects defined within the database, but stored
externally to the database.'' Thus they are implicitly disabled, which
fulfills the requirement.

\bydefault{RHEL5,RHEL6}{iacontrol}{DCFA-1}\bydefault{RHEL5,RHEL6}{databasestig}{DG0099}%
PostgreSQL as included in RHEL is prevented from running external
executables by the SELinux policy.

\notapplicable{iacontrol}{DCFA-1}\notapplicable{databasestig}{DG0101}%
PostgreSQL as included in RHEL is prevented from running external
executables by the SELinux policy; therefore no OS accounts are used to
``execute external procedures.''

\bydefault{RHEL5,RHEL6}{iacontrol}{ECLP-1}\bydefault{RHEL5,RHEL6}{databasestig}{DG0120}%
Since PostgreSQL as included in RHEL does not support external objects and
cannot run external executables, users inside the database are effectively
(if trivially) prevented from accessing ``objects stored and/or executed
outside of the DBMS security context.''

\bydefault{RHEL5,RHEL6}{iacontrol}{DCFA-1}\bydefault{RHEL5,RHEL6}{databasestig}{DG0102}%
All ``DBMS processes or services'' for PostgreSQL as included in RHEL are
owned by the \verb!postgres! user, which is a ``custom, dedicated OS
account.''

\bydefault{RHEL5,RHEL6}{iacontrol}{DCPA-1}\bydefault{RHEL5,RHEL6}{databasestig}{DG0111}%
All ``DBMS data files, transaction logs and audit files'' for PostgreSQL
as included in RHEL are stored in ``dedicated directories... separate from
software or other application files.'' These are under
\verb!/var/lib/pgsql!, and there are separate directories for each of the
three kinds of files. Permissions to each are ``customized to allow access only
by authorized users and processes.''

\bydefault{RHEL5,RHEL6}{iacontrol}{DCPA-1}\bydefault{RHEL5,RHEL6}{databasestig}{DG0112}%
DBMS system data files for PostgreSQL are ``stored in dedicated disk
directories.''

\doneby{DBAs}{iacontrol}{DCPA-1}\doneby{DBAs}{databasestig}{DG0113}%
To prevent ``database tables from unrelated applications'' from being
``stored in the same database files'' under PostgreSQL, ensure that for
each ``unrelated application'' there is a separate database, using the
\verb!createdb! utility as appropriate.

\doneby{admins}{iacontrol}{COBR-1}\doneby{admins}{databasestig}{DG0114}%
Make sure that ``DBMS files critical for DBMS recovery'' are ``stored on
RAID or other high-availability storage devices,'' by specifying a RAID
hard drive setup when procuring any server on which a PostgreSQL database
will reside.

\doneby{DBAs}{iacontrol}{ECLP-1}\doneby{DBAs}{databasestig}{DG0116}%
Do not grant ``DDL (Data Definition Language) and/or system
configuration'' privileges to non-privileged DBMS users. To obtain a
``list of privileged role assignments'' in an installation of PostgreSQL
as included in RHEL, perform the following commands as root on the server
in question:

\begin{verbatim}
    sudo -u postgres psql
    \l
    [A list of databases and privileges is output.]
    \du
    [A list of roles and privileges is output.]
    \c foo
    \dp
    [A list of objects and privileges is output.]
    \q
    #
\end{verbatim}

Replace `foo' in the above directions with the name of a database from the
list output by \verb!\l!. There may be multiple databases. This data is
sent to administrators automatically in a monthly report;
see~\S\ref{class_postgresql::stig}. 

See~\S\ref{DBDocumentedWithIAO} for the list of IAO-approved DBA role
assignments.


\bydefault{RHEL5,RHEL6}{iacontrol}{ECAN-1}\bydefault{RHEL5,RHEL6}{databasestig}{DG0123}%
Access to ``DBMS system tables and other configuration or metadata'' is
suitably restricted by default. See Chapter 44, ``System Catalogs,'' in
\cite{pgsql-documentation}.

\doneby{DBAs}{iacontrol}{ECLP-1}\doneby{DBAs}{databasestig}{DG0004,DG0124}%
Do not use a privileged database account for non-administrative purposes.
For each application in the database, create a per-application object
owner user and/or per-application administrator user; use one of these,
and not a DBA account, to create the objects necessary for the application
and to maintain the application. Disable this account ``when not
performing installation or maintenance actions.''

\doneby{DBAs}{iacontrol}{ECSD-1}\doneby{DBAs}{databasestig}{DG0015}%
For each application which uses the database, make sure that the database
users which are used in production are not allowed to execute DDL
statements (\emph{e.g.} creating and dropping tables, indices, views,
etc.).

\documents{iacontrol}{DCSS-1}\documents{databasestig}{DG0155}%
``Trustworthiness'' of ``data files and... configuration files'' for
PostgreSQL as included in RHEL is provided by the underlying operating
system. See~\S\ref{iac-DCSS-1} for a summary of measures taken to preserve
system state integrity.

\notapplicable{iacontrol}{ECLO-1}\notapplicable{databasestig}{DG0160}%
According to its documentation \cite{pgsql-documentation}, PostgreSQL does
not appear to provide a means to ``restrict the number of failed logins
within a specified time period.''

\notapplicable{iacontrol}{ECDC-1}\notapplicable{databasestig}{DG0170}%
A review of the PostgreSQL documentation \cite{pgsql-documentation}
indicates that there is no way to turn off transaction journalling in
PostgreSQL; thus it is enabled as required, but the checklist says, ``If
no configuration settings are available to enable or disable transaction
journaling, this check is Not Applicable.''

\doneby{DBAs}{iacontrol}{ECLP-1}\doneby{DBAs}{databasestig}{DG0063}%
Do not grant ``privileges to restore database data, objects, or other
configuration or features'' to unauthorized DBMS accounts. 

\documents{iacontrol}{ECML-1}\documents{databasestig}{DG0087}%
Because PostgreSQL as included in RHEL ``does not provide the capability
to mark or label sensitive data within the DBMS, this check is Not a
Finding.''

\documents{iacontrol}{ECLO-1}\documents{databasestig}{DG0133}%
PostgreSQL as included in RHEL ``does not provide a method or means for
configuration of account lock times,'' so ``this check is Not a Finding.''








\section{Database STIG compliance under SQLite}
\label{DatabaseSTIGSQLite}

SQLite is not a traditional server-based database. It is, quoting from its
website, ``a software library that implements a self-contained,
serverless, zero-configuration, transactional SQL database engine.''
Because it does not implement a client-server interaction model, it
doesn't listen over the network, nor authenticate users. Authorization to
operate on the database is based on operating-system-level access to the
single file containing the database, so there is no system of user
accounts with differing levels of access. SQLite also has no run-time
configuration. Consequently many Database STIG requirements cannot be
applied to SQLite. Those dealing with control of the files that comprise
an SQLite installation, including security patches, are applicable and are
covered in~\S\ref{dbms-with-rhel}.

\documents{iacontrol}{DCCS-1}\documents{databasestig}{DG0007}%
The Database STIG is the primary document used in securing the SQLite
DBMS, as far as it applies.

\bydefault{RHEL5,RHEL6}{iacontrol}{DCFA-1}\bydefault{RHEL5,RHEL6}{databasestig}{DG0014}%
SQLite as distributed in RHEL contains no ``demonstration or sample
data\-bases or applications.''





\section{Databases included with RHEL}
\label{dbms-with-rhel}

Some requirements are met by existing policies and procedures written
throughout this \CMITSPolicy\, and notated in those existing places. See
the Database STIG Compliance index for an exhaustive list of these; look
on the referenced pages for phrases like ``databases included with RHEL.''

\documents{iacontrol}{VIVM-1}\documents{databasestig}{DG0003,DG0097}%
For DBMSes included with RHEL, updates and patches are handled as for any
RHEL update. See~\S\ref{Patching}.

\documents{iacontrol}{DCSL-1}\documents{databasestig}{DG0009}%
Permissions for software libraries relating to DBMSes included with RHEL
are controlled by RHEL's packaging system, and are restricted to the
fewest accounts requiring access. 

\documents{iacontrol}{DCSL-1}\documents{databasestig}{DG0010}%
Permissions and changes to database executable and configuration files for
DBMSes included with RHEL are checked periodically by
\S\ref{class_rpm::stig} and~\S\ref{class_aide}.

\documents{iacontrol}{DCPR-1}\documents{databasestig}{DG0011}%
``Software libraries'' and ``management tools'' for DBMSes included with
RHEL are managed and patched using the same procedures as the operating
system software. See~\S\ref{Patching}.

\bydefault{RHEL5,RHEL6}{iacontrol}{DCPA-1}\bydefault{RHEL5,RHEL6}{databasestig}{DG0012}%
Data and configuration directories for DBMSes included with RHEL, where
applicable, are dedicated for those purposes by the operating system.
For executables and libraries, SELinux is the ``method that provides...
separation of security context.'' Access controls are well-defined through
the RPM packaging system, mitigating the discussed vulnerability.

\bydefault{RHEL5,RHEL6}{iacontrol}{DCFA-1}\bydefault{RHEL5,RHEL6}{databasestig}{DG0016}%
DBMSes included with RHEL have separate components in separate RPM
packages; unneeded components are not installed.

\bydefault{RHEL5,RHEL6}{iacontrol}{DCSL-1}\bydefault{RHEL5,RHEL6}{databasestig}{DG0019}%
For DBMSes included with RHEL, ownership of ``DBMS software libraries and
configuration files files'' is set by the vendor in the RPM package.

\bydefault{RHEL5,RHEL6}{iacontrol}{COTR-1}\bydefault{RHEL5,RHEL6}{databasestig}{DG0115}%
DBMS system files for DBMSes included in RHEL are provided on the OS
media; trusted recovery measures used for the OS apply to the DBMS
software as well.

\notapplicable{iacontrol}{ECWM-1}\notapplicable{databasestig}{DG0179}%
``The DBMS warning banner should meet DoD policy requirements,'' but ``a
warning banner displayed as a function of an Operating System or
application login for applications that use the database makes this check
Not Applicable.'' See~\S\ref{iac-ECWM-1} for summaries of where warning
banners are installed by this policy; per-application warning banners are
covered in per-application documentation.

\documents{iacontrol}{COSW-1}\documents{databasestig}{DG0187}%
``DBMS software libraries'' for DBMSes included in RHEL are part of the
operating system distribution, so OS install media contains them.
See~\S\ref{ProceduresCriticalSoftware} for procedures regarding OS install
media; see~\S\ref{iac-COSW-1} for other assurances about software needed
for operations continuity.






\section{Requirements implemented by each system}
\label{per-system-databasestig}

(``System'' here means an Automated Information System (AIS), not an
individual host.) The requirements not covered by this \CMITSPolicy ,
which must be addressed in per-system documentation, are summarized here:

\begin{description}
\item[DG0011] How database configuration files and stored procedures are
configuration-managed; how system personnel interface with IT regarding
database software patches
\item[DG0013] How the database is backed up and recovered, and evidence
that such procedures have been followed
\item[DG0017] Whether production and non-production databases reside on
the same host, and, if so, who authorized that
\item[DG0019] Ownership of ``application software and configuration
files'' (this may be covered in a system-specific way by a piece of policy
elsewhere in this \CMITSPolicy\, rather than in a system-specific
document)
\item[DG0020] How database backups are verified and backup procedures
tested, and evidence that such testing and verification procedures have
been followed
\item[DG0064] How backups are protected during all phases of backup and
recovery
\item[DG0065] How users are authenticated using DoD PKI certificates, or
how this requirement is mitigated
\item[DG0069,DG0076] How exported production data is protected and
modified, if or when it is imported into a development database
\item[DG0066, DG0067, DG0071, DG0072, DG0073, DG0079, DG0125, DG0126,
DG0127, DG0128, DG0129] Considerations regarding password authentication,
if it is used
\item[DG0078] A list of authorized DBMS accounts, and how each use of
those accounts is mapped to a specific person
\item[DG0088] How periodic and unannounced vulnerability scans of the
database are conducted
\item[DG0090,DG0092,DG0106] How sensitive data are encrypted at rest if required
\item[DG0093] How remote administrative database access can only happen
over encrypted channels
\item[DG0096] How DBMS IA policies and procedures are reviewed at least
once a year
\item[DG0103] How the DBMS server software is configured to limit access
by network address
\item[DG0104] How DBMS ``services/processes'' are named in a clearly
identifiable fashion. ``An example ... [is] \verb!prdinv01!.''
\item[DG0107] Identification of any ``sensitive data'' such as PII or
classified which is stored in the DBMS
\item[DG0108] ``Assignment of the priority to be placed on restoration of
the DBMS''
\item[DG0109] How the DBMS is isolated from other application services
\item[DG0110] How the DBMS is isolated from ``security support
structures'' such as Windows domain controllers or Kerberos servers
\item[DG0116] A list of IAO-approved roles ``assigned privileges to
perform DDL and/or system configuration actions''
\item[DG0118] How the IAM reviews changes to DBA role assignments
\item[DG0119] That the ``application user'' does not have ``administrative
privileges.''
\item[DG0124] Which DBMS accounts, specific to the application, create and
maintain the DBMS objects needed by the application
\item[DG0151] That the DBMS listens on a static, default port, if the DBMS
listens over the network
\item[DG0156] Who is the IAO for the DBMS, and evidence that the IAM has
assigned and authorized that person
\item[DG0157, DG0158, DG0159, DG0198] How remote database administration
is either disabled, or documented, authorized, audited, monitored by the
IAO or IAM, and done over an encrypted channel
\item[DG0167] How sensitive data served by the DBMS is encrypted in
transit
\item[DG0186] How the database is protected from access originating from
public or unauthorized networks
\item[DG0187] (Possibly) How to quickly reinstate operation of the
application that talks to the database, in case of contingency
\item[DG0075,DG0190,DG0191,DG0192] How the database talks to remote
databases and applications in a compliant and secure manner, if it does
\item[DG0089,DG0194,DG0195] How developers are kept from disturbing
production DBMS instances
\item[DG0008] A list of authorized object owner users in the application's
database
\item[DG0060] A list of ``non-interactive, n-tier connection, and shared
accounts,'' evidence of approval of these by the IAO, and how each action
taken by one of these users can be traced to an individual person
\item[DG0070,DG0074] User account lifecycle for database users, including
deletion
\item[DG0091] How ``custom and GOTS application source code stored in the
database'' has been ``protected with encryption or encoding''
\item[DG0105] Authorized list of privileges for application users
\item[DG0119] That application users do not have ``administrative
privileges'' such as creating tables and other DDL
\item[DG0121] How application user privileges are granted solely by
granting membership in roles, not directly to the application user
\item[DG0122] How access to ``sensitive data'' (such data as the
information owner would deem as requiring encryption) is restricted to
authorized users
\item[DG0138] How ``access grants to sensitive data'' (such as requires
encryption) are ``restricted to authorized user roles''
\item[DG0165,DG0166] How symmetric and asymmetric (resp.) keys are
protected in a compliant fashion (if data is encrypted in the database)
\item[DG0172] How changes to security labels are audited (if sensitive
data needing encryption or classified data are stored in the database)
\item[DG0193] How non-interactive account passwords expire at least every
year
%
\end{description}


See the Database STIG and security checklists for details on these
requirements.





\section{Requirements which may become applicable in future}
\label{DatabaseFutureRequirements}

\begin{description}
\item[DG0085] If a database administrator is needed in future, the
least privileges needed by that user for day-to-day operation must be
determined, and the user must be limited to those privileges.
\end{description}

\notapplicable{iacontrol}{DCPP-1}\notapplicable{databasestig}{DG0152}%
No PostgreSQL installations under the purview of this policy accept
connections across ``network or enclave boundaries as defined in the PPS
CAL'' at \url{http://iase.disa.mil/ports/index.html}.

\documents{iacontrol}{ECIC-1}\documents{databasestig}{DG0171}%
Because no DBMSes containing classified information are presently managed
by this policy, there are trivially no interconnections between DBMSes of
differing classification levels.

\documents{iacontrol}{DCFA-1}\documents{databasestig}{DG0100}%
Because no DBMSes using replication are presently managed by this policy,
no DBMS accounts exist for the purpose of replication.

\documents{iacontrol}{ECLO-2}\documents{databasestig}{DG0135}%
Because no DBMSes containing classified information are presently managed
by this policy, DBMS users need not be notified at login regarding
previous successful and failed DBMS login attempts.





\section{Requirements which need further attention}
\label{DatabaseFurtherAttention}

For PostgreSQL as included in RHEL:

\unimplemented{databasestig}{DG0052}{Names of applications that access the
database may not be logged in the audit trail.}

\unimplemented{databasestig}{DG0054}{Because names of applications may not
be logged, DBMS access using unauthorized applications may not be
discovered by monitoring the audit logs.}



\section{Things which must be documented with the IAO}
\label{DBDocumentedWithIAO}

\documents{iacontrol}{ECLP-1}\documents{databasestig}{DG0087,DG0116}%
The IAO-authorized list of ``roles assigned privileges to perform DDL
and/or system configuration actions in the database'' in PostgreSQL as
included in RHEL is this:

\begin{itemize}
\item \verb!postgres!
\end{itemize}

As the \verb!postgres! user cannot log in, only system administrators can
become this user.

Changes to this list must be discussed with the IAO, and changes are of
course tracked. Each AIS may also have a list of database administrative
roles.

\documents{iacontrol}{DCSD-1}\documents{databasestig}{DG0153}%
The IAO-authorized list of DBA role assignments in PostgreSQL as included
in RHEL is this:

\begin{itemize}
\item \verb!postgres!
\end{itemize}

As the \verb!postgres! user cannot log in, only system administrators can
become this user.

Changes to this list must be discussed with the IAO, and changes are of
course tracked. Each AIS may also have a list of database administrative
roles.

