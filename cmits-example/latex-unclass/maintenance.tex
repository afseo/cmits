% --- BEGIN DISCLAIMER ---
% Those who use this do so at their own risk;
% AFSEO does not provide maintenance nor support.
% --- END DISCLAIMER ---
% --- BEGIN AFSEO_DATA_RIGHTS ---
% This is a work of the U.S. Government and is placed
% into the public domain in accordance with 17 USC Sec.
% 105. Those who redistribute or derive from this work
% are requested to include a reference to the original,
% at <https://github.com/afseo/cmits>, for example by
% including this notice in its entirety in derived works.
% --- END AFSEO_DATA_RIGHTS ---
\chapter{Maintenance}
\label{Maintenance}

This chapter discusses how to maintain this policy, both as a set of
rules for computers to follow and as a document for humans to
read. We'll talk about how to build your own copy of this document; a
general approach to using policy-based tools to maintain a set of
systems; details you need to keep in mind as you maintain the policy
and this document; and what you would want to keep in mind if you were
to make CMITS over from scratch.



\section{How to build a copy of this document}
\label{generating}

First, obtain a copy of the document's sources. The \CMITSPolicy\ is
frequently stored and tracked in a Subversion repository. We'll say, for
example, that everything is under
\url{https://example.com/svn/trunk/myorg-cmits}. Check out a working copy
of the directory using your Subversion client. In your working copy
folder, you will find {\tt modules-*} and {\tt manifests} directories,
which contain the Puppet source code and other attendant files, and you
should find a {\tt unified-policy-document} directory. This directory
is where you can generate the policy document from the manifest stored in
the {\tt modules-*} and {\tt manifests} directories and the documentation
stored in the {\tt latex-*} directories. See the README.txt in the {\tt
unified-policy-document} directory for how to proceed.




\section{General process}

Here, in general, is how to maintain this policy. We'll use the word
\emph{problem} here to mean something that needs to be changed. Think of
it like a word \emph{problem}, not like a drinking \emph{problem}.

\begin{description}

\item[A problem appears:]
A new security requirement comes down, or a user can't run a program. The
machine as configured by the policy does not do what is needed.

\item[Relate the problem to configuration:]
How does the configuration of the machine bring up the problem? Is a wrong
directory on the path, does a package need to be installed?

\item[Express the solved state:]
With the problem solved, what's different about the system? Is there an
extra line in a file? A certain version of a package installed? Does a
file have different permissions? That end state is what you will express
with Puppet, not so much the exact steps needed to get there.

\item[Have a policy working copy:]
Check out, if necessary, a copy of the policy from the Subversion server.

\item[Locate the configuration and relate it to the policy:]
Think about what subsystem needs to be configured. Each \emph{module} in
the {\tt modules} directory deals with a subsystem, e.g. {\tt ssh}, {\tt
nfs}. Find or create the module to which your configuration belongs. Each
module contains \emph{manifests}, files which contain \emph{classes},
which in turn contain enough \emph{resources} (the individual units of
configuration) to express a single goal. For example, {\tt
ssh::no\_tunnelling} is a class which turns off all tunnelling of network
connections and X11 traffic through SSH sessions.

\item[Change, write, or co-opt classes that change the configuration:]
If you write classes, use other modules as examples, and Puppet
reference documentation as a resource. When writing, keep in mind that you
have four audiences: Puppet, which will be implementing the policy; other
administrators, who need to read and understand the policy; your future
self, six months down the road; and auditors. See below for more details
about how to write for each of these audiences.

The Puppet community has a set of common modules called the Puppet Forge;
if you use one, take intellectual rights into consideration, be sure you
know what other modules it depends on, and count on re-documenting it: the
CMITS documentation scheme, for better and for worse, serves more purposes
than puppetdoc does.

If you change a module, be sure you know where in the policy it is used:
you may be reconfiguring more hosts than you think.

\item[Change manifests to include your classes:]
On what nodes, or hosts, does the change need to happen? All hosts which
are to be compliant with the requirements of a document (like a STIG)? All
hosts in a given room? All hosts belonging to a given subgroup of the
organization?  Find or create a suitable class in {\tt
manifests/templates.pp}; modify {\tt manifests/nodes.pp} if necessary to
make some hosts include your new class.

\item[Test:] Use \textsf{rspec-puppet} to test everything about your
  module that you can. Such tests can be easily automated and are
  saved with the code. The quickest way to make sure your module does
  what you want on your own host is to use \verb!puppet apply!
  something like so:
\begin{verbatim}
sudo puppet apply <<< 'include mynewclass'
\end{verbatim}
  Then if it didn't work right, manually restore whatever system
  settings were changed and try again.

\item[Manage changes:]
Use a software version control system to track changes to the policy. This
helps answer questions of why a change was made later on, and ensures that
your changes are properly backed up and deployed.

\end{description}




\section{Invariants}
\label{invariants}

As you maintain the policy, there are several important properties of it
that you must maintain. 

\begin{description}

\item[Self-documenting:]
Write everything you know about the aspect of the configuration that your
policy changes. See \S\ref{class_fips} (as of revision 4597, 1 Nov 2011)
for a good example. This property makes the policy document mean something
to human administrators (including your future self), both during
production and in a contingency situation. It also makes the policy
document a central place for small but important facts about quirks of the
subsystems being configured.

\item[Discoverable:]
Not only the policy files themselves, but also the policy document and its
history should be easy places to search for needed knowledge. Take the
time to write a cross-reference to another section of the policy, a
bibliographical citation to another document containing guidance, the
official number of a controlled requirement, a revision number in the
admin repository when something was fixed or broken. Links like these made
the World Wide Web the amazing resource it is.

\item[Flexible:]
To the greatest extent possible, the policy should not write over changes
not under its control. For example, \S\ref{class_mail::stig} edits
Postfix's configuration, rather than copying an entirely new configuration
file over the old one. If an updated postfix package is issued because of
a security update, and it changes the Postfix configuration slightly in an
unrelated area, the policy that edits the file will still work against
updated machines, while a policy that copied over the file would miss
something.

\item[Authoritative:]
Any change that needs to be made to any system should be part of the
policy. This property is what makes contingency recovery using this policy
so easy, and what makes this policy document as complete as it is.

\item[Managed:]
Every change you make should be checked into the version control
repository, under your name. This eases compliance with audit-related
regulations, and plays into the automated policy updating and backup
that's part of the policy (\S\ref{module_backup}).

\item[Convergent:]
The thing that lifts Puppet above shell scripting is that when you use its
elements to write your policy, you gain the guarantee that a managed host
will always move toward conformance with the policy. If you write a shell
script, you have to make sure it's \emph{idempotent} (running it multiple
times has the same effect as running it once), and that it deals with all
possible errors and unexpected inputs.

\end{description}







\section{How to write this document}
\label{writing}

Any line in a {\tt *.pp} file which starts with a pound sign (\#) will be
fed to {\LaTeX} when the documentation is built. This is by means of {\tt
shaney}, which strips the comment characters off, and surrounds uncommented
Puppet policy code with verbatim tags so that it will be typeset as code,
and so that {\LaTeX} will not search it for markup tags. The outputs of
shaney for each file are concatenated in a certain order.

{\tt shaney} also constructs the
\S\ref{iac-compliance-summary}~(\nameref{iac-compliance-summary}) chapter.

Here's what this means for you, the documentation writer:

\begin{itemize}

\item If you put an underscore (\_) in a comment, put a backslash
(\textbackslash) before it so that {\LaTeX} will not barf.

\item Comments with whitespace before the \# character are typeset as
  code; comments starting on the first column are treated as
  discussion. If you comment something out, kindly put spaces before
  the \# characters, so that your commented-out policy won't be
  treated as text. By the same token, if you write a comment about
  some nicety of Puppet syntax you used, and not about what the policy
  is, you may want to indent it.

\item In any module, the \verb!init.pp! comes first, then other
\verb!*.pp! files in the same directory in alphabetical order, then
\verb!*.pp! files in subdirectories in alphabetical order. So you should
start the \verb!init.pp! with {\tt \# \textbackslash
section\{\emph{Subsystem name}\}}; start other \verb!*.pp! files with a
subsection directive, and \verb!*.pp! files in subdirectories with a
subsubsection directive, so that the structure of the finished document
mirrors the directory structure of the module.

\item If you write \verb!\S\ref{class_other::class}! in the comments
  of your file, readers of the raw text of the file will know to look
  at \verb!modules/other/manifests/class.pp!; when the document is
  typeset, the reference will turn into a hyperlink to the section
  number where the class is written.

\item When you write an implements tag
  \verb!\implements{iacontrol}{FOO-1,BAR-1}!, all lines from that line
  to the next paragraph break or to the next piece of Puppet code will
  go into the \nameref{iac-compliance-summary} chapter. So aim that
  first paragraph toward auditors: use language familiar to them by
  quoting the requirement; don't go into detail about the policy, or
  things you found out while configuring the system properly; and
  don't say anything funny or offer personal opinions. Write details
  and opinions in ensuing paragraphs.

\item There's a {\LaTeX} cheat sheet at
\url{http://www.stdout.org/~winston/latex/}.

\item \index{g}{SELinux} Changes to SELinux are usually deployed in
\emph{policy packages}, which are files whose names end with \verb!.pp!.
If you store any of those files within this policy, you must make sure
that the name of the file inside the policy ends not just with \verb!.pp!
but with \verb!.selinux.pp!. That way, the scripts that build the unified
policy document will know that such files do not contain Puppet code and
\LaTeX\ comments, but SELinux policy.

\item Write only plain text in section or chapter names: no markup,
  such as \verb!\emph! or \verb!\tt!. Normally \LaTeX\ supports this,
  but the hack which automatically writes names of pertinent IA
  controls after section names in the table of contents is brittle,
  and causes \LaTeX\ to fail when you do this.

\end{itemize}




\section{How it all works}
\label{workings}

The building of this document is done by \verb!sourapples!, which is a
part of \verb!shaney!. \verb!sourapples! first generates all the generated
parts of the document, then calls \LaTeX , \verb!makeindex!, and other
utilities, to typeset the document.


\subsection{Written \LaTeX\ parts}

The main document is \verb!main.tex!. This sets the title of the document,
pulls in the \LaTeX\ packages used, and includes each chapter of the
document in order.

Prose chapters and document parts are included from the \verb!latex-fouo!
(if it exists) and \verb!latex-unclass! directories.

Some chapters are not written, but generated from many smaller files.
These are the generated parts.

\subsection{Generated parts}
\label{workings:generated}

The Puppet policy is stored in the \verb!*.pp! files in the
\verb!manifests!, \verb!modules-unclass! and perhaps \verb!modules-fouo!
directories. Shaney finds them all, removes comment characters and
surrounds Puppet code with verbatim tags, resulting in the {\tt
policy.tex} file. During this process it generates the index directives
that result in the indices of classes, defined resource types, and files.
It also pulls out per-IA-control excerpts using the \verb!\implements!
tags. The documentation in the Puppet code is pulled together into the
``Policy'' chapter; the excerpts comprise the ``Compliance by IA control''
chapter.

The attendant files are in the \verb!modules-*/*/files! directories.
\verb!sourapples! gathers them and converts the ones which seem to be made
of readable text into a form suitable for inclusion into the policy
document. The result of this is the ``Attendant files'' chapter.



\section{Document requirements}

If you were to transition this document to another document preparation
system, you would need to re-engineer it from its requirements, and so you
would need to know those requirements.

Guiding principles for the policy are outlined in~\S\ref{invariants}.
Guiding principles for this document are given in the Colophon
(\S\ref{Colophon}).

Sources of requirements for this document:

\begin{itemize}
\item We are administering computers every day with the contents of
  this document, and functional requirements on their configuration
  change every day. To successfully document this, our documentation
  must be primarily organized in the same way our problems and
  configuration changes are.
\item We are submitting this document to other organizations to back
  up our claims of compliance with several \emph{requiring documents}
  (for example, the UNIX SRG). Those other organizations don't have
  time to read our whole document.
\item In case of contingency we may need to read directly how systems
  should be configured, rather than delegating the task of configuring
  them to a tool.
\item Several \emph{requiring documents} (for example, the UNIX SRG)
  place \emph{named requirements} on the configuration of our
  computers or our procedures. We need to know what our expected
  compliance posture is, \emph{i.e.}, the set of named requirements
  met when the policy is applied, plus the set of reasons why unmet
  requirements are unmet. The \emph{requiring documents} may change a
  few times per year; corresponding changes to our policy may be
  needed.
\item We are making a document inside the DoD.
\end{itemize}

Requirements:

\begin{enumerate}
\item The parts of the document containing the policy must be
programmatically constructed from comments written in the policy.

\item Other parts containing prose (such as the part you are reading right
now) must also be integrated into the document.

\item Supporting files, such as configuration files copied into place by
the policy, should also be included in the document.

\item It must be easy to notate our posture as regards \emph{named
requirements}, such as IA controls and requirements from multiple
STIGs---both in comments in policy files and in prose sections. The
postures regarding compliance at the time of this writing are:
    \begin{itemize}
    \item this piece of the policy automates compliance
    \item we are not yet compliant
    \item compliance comes through the action of some people, like
    administrators, or users
    \item the default configuration of an operating system or piece of
    software we use is compliant
    \item the requirement is not applicable
    \item the requirement is to document something, and here is the
    documentation of that thing
    \end{itemize}

\item It must be easy to see whether a piece of the document has to do
with a named requirement, which one, and what the posture is. For example,
a compliance notation could result in a margin note in the document, which
is red if we are not compliant.

\item It must be easy to find all parts of the document relating to a
given requirement, and what posture they put us in. For example, each
compliance notation could result in an entry in a per-requiring-document
index, notating that the requirement is ``automated,'' or ``N/A.''

\item It should be easy to find all parts of the policy relating to a
given file, class or defined resource type.

\item Where one part of the policy refers to another (\emph{e.g.}, a class
includes another class) there should be a quick way to get to the
corresponding part of the document, like a clickable link.

\item There must be a way to get quickly to a given section of the
document, for example a table of contents, or if the output file is a PDF,
PDF bookmarks pointing to each section.

\item Along with the name of each section in the table of contents, there
should be a list of the IA controls dealt with in that section.

\item A summary of compliance broken out by requiring document, in CSV
(Comma-Separated Value) format or a similarly easy-to-parse format, must
be derived from the compliance notations---including short prose reasons
for non-compliance. (CSV may not be appropriate for the prose.)

\item There must be a chapter which summarizes compliance with IA
controls, sorted by IA control. It must be programmatically generated. It
should provide a quick way to get to the detailed parts of the document
relating to each IA control.

\item A given compliance posture notated with regard to a STIG
requirement, where the STIG details IA controls related to each STIG
requirement, must be programmatically interpreted as the same posture with
regard to the corresponding IA controls, and summarized in the
per-IA-control chapter as such.

\item Security labels must be written at the top of every page.

\item The title page must contain a security label, the title, the date,
the organizational logo, a distribution statement and a destruction
notice.

\end{enumerate}


